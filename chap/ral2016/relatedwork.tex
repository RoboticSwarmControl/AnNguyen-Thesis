

This chapter describes research performed for the IEEE RA-L Paper ``title'', by Srikanth...
My contribution was the .... and the ... and the.... .



\section[RelatedWork]{Overview and Related Work}

This paper presents a \emph{heterogeneous sensor system} for automatic sensor deployment. The goal is to overcome the drawbacks of manually deploying seismic sensors. 
In previous work \cite{sudarshan2015using}, we demonstrated a UAV equipped with four geophone sensors as landing gear.
This UAV automated sensor deployment by flying to GPS waypoints to obtain seismic data. 
This paper's main contributions are to implement a heterogenous seismic sensing team, to perform a small-scale seismic survey, provide experimental data on geophone plant quality as a function of drop height and soil type, and present a tool for analyzing the survey logistics.
Our multi-agent system approach is designed to quickly and efficiently perform a survey.

The geophones in  \cite{sudarshan2015using} were connected to the UAV, causing four problems:
(1) a UAV was required for each additional sensor,
(2)  the force for planting the geophone was limited by the weight of the UAV,
(3) the platform required a level landing site,
(4) the magnets in the geophones distort compass readings, causing landing inaccuracy when autonomous.

The proposed heterogeneous sensor system separates the sensing units from the UAV, reducing the cost per sensor. 
Our first solution uses \emph{SeismicDarts}, geophones dropped by the UAV.
Dropping the geophones enables increasing geophone penetration by increasing drop height and eliminates the necessity for a level landing site.
The new design also increases the separation between geophones and the UAV.

However, some survey locations are in regions inaccessible to UAVs such as forests, thin atmosphere environments, caves, or hard surfaces which the SeismicDart cannot penetrate. These challenging regions motivate the development of a walking ground robot, the SeismicSpider. 
 The SeismicSpider can be deployed by a UAV, but is a mobile platform with autonomy and sensing that enable it to move to desired locations and conduct seismic measurements.

\subsection{Overview of Seismic Sensing Theory}

\begin{figure}
\centering
\begin{overpic}[width=\columnwidth]{ral2016/Overview.pdf}\end{overpic}
\caption{\label{fig:sensor_types}
 Comparing state-of-the-art seismic survey sensors. a.) A traditional cabled system connects geophones in series to a seismic recorder and battery. b.) Autonomous nodal systems give each geophone a seismic recorder and battery.}
 \vspace{-2em} 
\end{figure}



During seismic surveys, a source generates seismic waves that propagate under the earth's surface. 
These waves are sensed by geophone sensors and recorded for later analysis to detect the presence of resources. 
Fig.~\ref{fig:sensor_types} illustrates the components of current sensors. 

\subsubsection{Geophones}
Magnet-coil geophones contain a permanent magnet on a spring inside a coil. Voltage across the coil is proportional to velocity. 
 Beneath the coil housing is a metal spike. 
  Geophones are \emph{planted} by pushing this metal spike into the ground, which improves coupling with the ground to increase sensitivity. 
 The magnet-coil must be vertical. 
  Misalignment reduces the signal proportional to the cosine of the error.


\subsubsection{Cabled Systems}
Hydrocarbon exploration extensively uses traditional \emph{cabled systems} for seismic data acquisition.
   Geophones are connected to each other in series using long cables. This cable is then connected to a seismic recorder and battery. 
    The seismic recorder consists of a micro-controller which synchronizes the data acquired with a GPS signal and stores the data onboard. 
 This method of data acquisition requires many manual laborers and a substantial expenditure for transporting the cables. 
 Rugged terrain makes carrying and placing cables labor intensive, and the local manual labor pool may be unskilled or expensive.
   
 \subsubsection{Autonomous Nodal Systems}
  \emph{Autonomous nodal systems}~\cite{wood1998distributed} are now being used to conduct seismic surveys. Unlike traditional cabled systems, autonomous nodal systems are not connected using cables. The sensor, seismic recorder, and battery are all combined into a single package called a \emph{node} that can autonomously record data as shown in Fig.~\ref{fig:sensor_types}. Even in these systems the data is generally stored in the onboard memory and can only be acquired after completing the survey. This delay means errors cannot be detected and rectified while conducting the survey. 
  Wireless autonomous nodes are a recent development. These systems can transmit data wirelessly in real time~\cite{jiang2015geophysical}. However, these systems still require manual laborers for planting the autonomous nodes at specific locations and deploying the large antennas necessary for wireless communication.
 
\subsection{Related Work}

Seismic surveying is a large industry.
The concept of using robots to place seismic sensors dates to the 1980s, when mobile robots placed seismic sensors on the moon~\cite{LSisMSE81}. \cite{DSSMaA14} and \cite{coste2013seismic} proposed using a mobile robot for terrestrial geophone placement. Plans are underway for a swarm of seismic sensors for Mars exploration~\cite{MAPL2006}.
Additionally,~\cite{muyzert2015marine} and~\cite{postel2014drone} proposed marine robots for hydrophone deployment underwater. 
Other work  focuses on data collection, using a UAV to wirelessly collect data from multiple sensors~\cite{wilcox2013seismic}. Autonomous sensor deployment and mobile wireless sensor networks were studied in~\cite{howard2002mobile,corke2004autonomous,tuna2014autonomous}. Heterogeneous mobile robotic teams were used for mapping and tracking in~\cite{howard2006experiments}.
% Their idea is to have a propulsion module that either be a wheels, tracks, turbines, helicopter blades etc. 
% This robot is used to deploy sensors at specified locations. 
% The system consists of modules for ensuring the sensor is well planted and placed perpendicular to the ground. The modular approach to a deployment system is innovative but might not be practically feasible. 
% While servicing a large area ability to deploy multiple sensors in limited time is a key factor. 
% The approach seems to be well suited for short surveys. 

%Some works relate to surveying an earthquake prone region to collect data as in \cite{dominici2012micro}. 
%The goals of these papers is to perform a survey rather than sensor deployment.
% An interesting proposal was to perform a seismic survey using augmented reality \cite{jones2016seismic}.
%\todo{
%\subsection{Sensor networks}
%\subsection{Multi-Robot Assignment}}
