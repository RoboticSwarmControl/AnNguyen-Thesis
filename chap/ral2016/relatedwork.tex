\section[Related Work]{Overview And Related Work}

This chapter presents a \emph{hetoerogenous seismic sensing system}, composed of stand alone geophone nodes deployed from unmanned aerial vehicles {UAV} and autonomous rovers with geophone attached.
We examine experimental data on geophone and soil coupling as a function of drop height and soil type.
We then provide a software tool for analyzing and planning a survey mission's logistic.
Our \emph{heterogenous sensor system} approach is designed to quickly and efficiently perform a survey with minimal manual labor for deployment and collection.

Sudarshan et al. \cite{sudarshan2015using} demonstrated a UAV equiped with four geophone sensors as landing gear.
The UAV in \cite{sudarshan2015using} can fly to a pre-programmed waypoint and land, attaching the geophones to the soil.

The geophones in  \cite{sudarshan2015using} had four problems:
(1) a UAV was required for each additional sensor,
(2) the force for planting the geophone was limited by the weight of the UAV,
(3) the platform required a level landing site,
(4) the magnets in the geophones distort compass readings, causing landing inaccuracy when autonomous.

The \emph{SeismicDart} presented in this system eliminates the need for a seperate UAV per sensor node.
Dropping the SeismicDarts from height also allows for greater penetration, firmer coupling and does not need a level landing site.
The new deployment unit also increase the distance between the SeismicDart's magnet and the UAV's magnetometer unit.

The SeismicSpider, our autonomous rover, can travel to survey locations inaccessible to UAV such as forests, thin atmosphere environments, caves or hard and rocky grounds which SeismicDarts cannot penetrate.
The SeismicSpider can also be deployed by a UAV, as close to the survey node as possible, then move to the desired location.

\subsection{Overview Of Seismic Sensing Theory}

\begin{figure}
	\centering
	\begin{overpic}[width=\columnwidth]{ral2016/Overview.pdf}\end{overpic}
	\caption{\label{fig:sensor_types}
	 Comparing state-of-the-art seismic survey sensors. a.) A traditional cabled system connects geophones in series to a seismic recorder and battery. b.) Autonomous nodal systems give each geophone a seismic recorder and battery.}
	%\vspace{-2em} 
\end{figure}



During seismic surveys, a source generates seismic waves that propagate under the earth's surface. 
These waves are sensed by geophone sensors and recorded for later analysis to detect the presence of resources. 
Fig.~\ref{fig:sensor_types} illustrates the components of current sensors. 

\subsubsection{Geophones}
Magnet-coil geophones contain a permanent magnet on a spring inside a coil. Voltage across the coil is proportional to velocity. 
 Beneath the coil housing is a metal spike. 
  Geophones are \emph{planted} by pushing this metal spike into the ground, which improves coupling with the ground to increase sensitivity. 
 The magnet-coil must be vertical. 
  Misalignment reduces the signal proportional to the cosine of the error.


\subsubsection{Cabled Systems}
Hydrocarbon exploration extensively uses traditional \emph{cabled systems} for seismic data acquisition.
Geophones are connected to each other in series using long cables. This cable is then connected to a seismic recorder and battery. 
The seismic recorder consists of a micro-controller which synchronizes the data acquired with a GPS signal and stores the data onboard. 
This method of data acquisition requires many manual laborers and a substantial expenditure for transporting the cables. 
Rugged terrain makes carrying and placing cables labor intensive, and the local manual labor pool may be unskilled or expensive.
   
\subsubsection{Autonomous Nodal Systems}
\emph{Autonomous nodal systems}~\cite{wood1998distributed} are now being used to conduct seismic surveys.
Unlike traditional cabled systems, autonomous nodal systems are not connected using cables.
The sensor, seismic recorder, and battery are all combined into a single package called a \emph{node} that can autonomously record data as shown in Fig.~\ref{fig:sensor_types}.
Even in these systems the data is generally stored in the onboard memory and can only be acquired after completing the survey.
This delay means errors cannot be detected and rectified while conducting the survey. 
Wireless autonomous nodes are a recent development.
These systems can transmit data wirelessly in real time~\cite{jiang2015geophysical}.
However, these systems still require manual laborers for planting the autonomous nodes at specific locations and deploying the large antennas necessary for wireless communication.
 
\subsection{Related Work}

Seismic surveying is a large industry.
The concept of using robots to place seismic sensors dates to the 1980s, when mobile robots placed seismic sensors on the moon~\cite{LSisMSE81}.
\cite{DSSMaA14} and \cite{coste2013seismic} proposed using a mobile robot for terrestrial geophone placement.
Plans are underway for a swarm of seismic sensors for Mars exploration~\cite{MAPL2006}.
Additionally,~\cite{muyzert2015marine} and~\cite{postel2014drone} proposed marine robots for hydrophone deployment underwater. 
Other work  focuses on data collection, using a UAV to wirelessly collect data from multiple sensors~\cite{wilcox2013seismic}.
Autonomous sensor deployment and mobile wireless sensor networks were studied in~\cite{howard2002mobile,corke2004autonomous,tuna2014autonomous}.
Heterogeneous mobile robotic teams were used for mapping and tracking in~\cite{howard2006experiments}.
