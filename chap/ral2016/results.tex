\section[Conclusion]{Future Work}

This chapter presented a \emph{heterogeneous sensor system} and technique for autonomous geophone deployment.
The \emph{heterogeneous sensor system} compose of two components, UAV deployable SeismicDarts, mobile SeismicSpider.
The work in this chapter allow us to automate tasks that currently require a much more manual labors in hazardous environment.

The SeismicDart's output is comparable to well-planted geophones. 
For hard surfaces where the SeismicDart could not penetrate, we presented an autonomous alternative, the SeismicSpider.  
The SeismicSpider is mobile, can actively adjust its sensors to ensure ground contact and vertical placement, and can be deployed and retrieved by UAVs.

Autonomous deployment was conducted using GPS, proving human involvement could be minimized by adopting the proposed technique.
Hardware experiments compared the autonomous system to manual planting and ballistic deployment.
Simulation studies show time and cost savings over traditional manual techniques.

Future systems should be weatherized and optimized for cost, robustness, range, and speed.
Soil maps could be used to plan a survey, allocating SeismicSpiders to rocky or forested areas and SeismicDarts to penetrable soils.
These maps can be made more accurate using drone-carried ground penetrating radar \cite{merz2015new}.
Alternatively, the  SeismicDart's internal accelerometer also provides feedback on the quality of the plant.
As shown in Fig.~\ref{fig:AnglePlotIndoors}, angular deviations indicate a higher drop height is needed.
