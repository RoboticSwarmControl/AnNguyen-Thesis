\section[Comparision]{Comparision}


\begin{figure} \centering
	{\includegraphics[width=\columnwidth]{ral2016/CannonPicture}}
	\caption{
		A pneumatic launcher for SeismicDarts.
		Ballistic dart deployment has limited usefulness because the incident angle is equal to the firing angle.} 
	\label{fig:CannonPicture}
\end{figure}

\begin{figure*}[htb]
	\centering 
	%\vspace{1em}
	%\renewcommand{\figwid}{0.48\columnwidth}
	\begin{overpic}[width =\figwid]{ral2016/sim1_1.pdf}
	\end{overpic}
	\begin{overpic}[width =\figwid]{ral2016/sim1_2.pdf}
	\end{overpic}
	\begin{overpic}[width =\figwid]{ral2016/sim1_3.pdf}
	\end{overpic}
	\begin{overpic}[width =\figwid]{ral2016/sim1_4.pdf}
	\end{overpic}
	\caption{
		Screenshots of simulations that were performed to estimate time take by different sensors surveying 100x100 m grid:
		a.) only SeismicSpiders
		b.) SeismicDarts and deployment system
		c.) heterogeneous system
		d.) human workers.
	\label{fig:Sim_overview}}
\end{figure*}

\subsection{Ballistic Deployment}
To compare an alternative deployment mechanism we built the pneumatic cannon shown in Fig.~\ref{fig:CannonPicture}a.
The pneumatic cannon is U-shaped,  2 m in length, with a 0.1 m (4 inch) diameter pressure chamber and a 0.08 m (3 inch) diameter firing barrel, connected by an electronic valve (Rain Bird JTV/ASF 100).
The cannon is aimed by selecting an appropriate firing angle $\theta_f$, azimuth angle, and chamber pressure.
The reachable workspace is an annular ring whose radius $r$ is a function of the firing angle and initial velocity $v$.
Neglecting air resistance, this range is found by integration:
\begin{align}
r = \frac{v^2}{g} \sin( 2 \theta_f ).
\end{align} 
Initial velocity is limited by the maximum pressure and size of the pressure chamber.
The cannon used  SCH 40 PVC, which is limited to a maximum pressure of 3 Mpa (450 psi).

We charged our system to 1 Mpa (150 psi), and achieved a range of $\approx$ 150 m.
This range is considerably smaller than the UAV's range, which when loaded can complete a round trip of $\approx 1.5$ km.

A larger problem, illustrated in Fig.~\ref{fig:CannonPicture}, is that angle of incidence $\theta_i$ is equal to the firing angle $\theta_f$.
Maximum range is achieved with $\theta_f = 45^\circ$, but this angle of incidence reduces the geophone sensitivity to $\cos(\theta_f )\approx 0.7$.
The placement accuracy of the cannon is lower than the UAV because a fired dart must fly over a longer distance than a dropped dart.
Safety reasons also limit applications for a pneumatic launcher.

\begin{table} \centering
	{\includegraphics[width=\columnwidth]{ral2016/simulation_table.pdf}}
	\caption{Comparison of different  deployment modes highlights the efficiency of UAV deployment.} 
	\label{tab:Sim_table}
\end{table}

\begin{figure} \centering
	{\includegraphics[width=\columnwidth]{ral2016/DronevsTime.pdf}}
	\caption{Survey time for a 1km x 10 km region for different numbers of UAVs.} 
	\label{fig:DronevsTime}
	%\vspace{-1em}
\end{figure}

\begin{figure} \centering
	{\includegraphics[width=\columnwidth]{ral2016/het_sen_ratio.pdf}}
	\caption{
		Survey time for different sensor ratios.
		The total number of sensors \{5000, 3000\} were kept constant.
		Ten darts were provided for each UAV.} 
	\label{fig:het_sen_ratio}
	%\vspace{-1em}
\end{figure}

\subsection{Simulation Studies}
A scheduling system to compare  time and costs for seismic surveys with varying numbers of UAVs, SeismicSpiders, SeismicDarts, and human laborers was coded in  {\sc Matlab}, available at \cite{Srikanth2016seismicScheduler}.
In each simulation, a seismic source must be measured at every survey point.
The scheduler must assign each sensor (SeismicDart or SeismicSpider) to an unmeasured survey point and assign each UAV or human worker a dart to pickup or deploy.
Once a sensor reaches a survey point, that sensor must wait until a seismic source is measured.
A vibration truck (blue) provides the seismic source.
Motion planning uses a centralized, greedy strategy.

Frames from four different cases on a small survey region are shown in Fig.~\ref{fig:Sim_overview} on a 100x100 m area with survey points at 10 m spacing:
a.) simulates 10 SeismicSpiders;  
b.) simulates 5 SeismicUAVs deploying 50 SeismicDarts.
Each UAV was allowed to carry up to 4 darts;
c.) simulates 10 SeismicSpider, 5 SeismicUAVs and 50 SeismicDarts;
d.) simulates 5 human workers deploying 75 SeismicDarts.
Each worker was allowed to carry up to 10 darts.
Survey points are grey circles if unmeasured and green if measured.
The simulation uses red hexagons for SeismicSpiders,  black diamonds for UAVs,  inverted yellow triangles for SeismicDarts, and magenta diamonds for human workers.
The assigned motion path for each sensor is colored magenta and the path completed is blue.

This tool allows us to examine engineering and logistic trade-offs quickly through simulations.
For example, Fig.~\ref{fig:DronevsTime} assumes a fixed number of darts and examines the finishing time with $5$ to $500$ UAVs.
The time required decays asymptotically, but $140$ UAVs requires only twice the amount of time required for $500$ UAVs, indicating  $140$ UAVs are sufficient for the task.   
 Substantial cost savings can be obtained by selecting the number of UAVs required to complete within a certain percentage greater than the optimal time.

The tool is useful for comparing the effectiveness of heterogeneous teams.
Table~\ref{tab:Sim_table} compares surveying a $1$ km x $10$ km strip of land with teams of (a) $5000$ SeismicSpiders, (b) $500$ UAVs and $5000$ SeismicDarts, (c) $500$ humans and $5000$ geophones.
Team (b) completed six times faster than team (c).
In Fig.~\ref{fig:het_sen_ratio}, the total number of mobile agents are constant, but the percentage of UAVs and SeismicSpiders are varied.
10 SeismicDarts were provided for each UAV.
Increasing the percentage of UAVs lowers the deployment time because UAVs move 20 m/s but SeismicSpiders move 0.2 m/s.
The velcity difference makes UAV deployment time-efficient.
