% Thesis 

\chapter[Conclusion]{Conclusion}
\label{chap-conc}

In this thesis, three applications for autonomous unmanned vehicles were presented.
Unmanned aerial vehicles (UAV) can be used to deploy, retrieve geophones for seismic surveying and deploy unmanned rover attached with geophones.
We presented software to plan for a wireless sensor network deployment using UAV and unmanned rover.
A UAV was used to drag an electrified net through an area to destructively survey mosquito populations.
We used our collaborator's algorithm to reduce the energy expenditure while maximizing area covered.
An updated design for drifting wireless sensor node was presented.
The drift node can be used to survey coastlines and national maritime boundaries for marine border security, wildlife science missions.

In future work, the applications above could be merged together for a more complete package.
A wireless sensor network can then be deployed, monitored and retrieved by AUVs.
Before the WSN deployment, the area can be surveyed by a UAV with minimal energy expenditure.
Wireless sensor nodes can communicate with each other to relay position, cache data or relay data back to the base station.
Like in \cite{sudarshanwsn}, UAVs can monitor data and charge wireless sensor nodes, extending their longevity.
For destructive mosquito surveying methods, the UAVs can incorporate live data to plan its route reactively.

%We developed a method to deploy geophones for seismic surveying with AUVs, reducing manual labors and injury risks.
%The geophones are combined with a data recording unit, can record its own position and the soundwave during the seismic survey.
%This eliminate long analog data transmisison cables, which reduce a large part of equipment weight during surveying.
%We can drop geophones from our UAVs when we put the geophones a SeismicDart shell, which eliminates the need for a separate UAV per sensor.
%For terrains that UAVs cannot fly to or the SeismicDart cannot approach, we developed a SeismicSpider, with geophones for legs.
%The SeismicSpider can travel to its desired position, with geohphones contacting the ground, record the soundwave during the survey.
%The software we developed to schedule a seismic survey can be used for many other wireless sensor network applications that use UAVs and autonomous rovers.
