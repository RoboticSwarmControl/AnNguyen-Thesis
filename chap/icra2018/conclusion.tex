\section{Conclusion and Future Work}\label{sec:conclusion}

This chapter presented an approach for finding optimal tours given turn costs and an energy budget, inspired by a mosquito-killing UAV with limited battery life. 
Initial experiments with the UAV and electrified screen track the location of a mosquito-killing UAV as it patrols a field and maps mosquito kills.  

%Future work
Many refinements to the algorithm could be pursued in future work, including changes to both the mosquito-biasing algorithm and the robot flight simulation.  The model may be expanded to continuous space, three dimensions, and to arbitrary turn angles.  These and other considerations will make a more realistic model for future work.  

Further testing of the multi-copter UAV is indicated and will allow more extensive testing of the robustness and accuracy of the hardware design. New sensors that can identify and detect flying insects~\cite{chen2014flying} may be added to the UAV and enable it to proactively steer toward insect swarms and identify insects in realtime.

The concept may be extended to a non-destructive population survey in which the screen could be replaced with a net and, with appropriate lighting, the camera used to record capture events.  Teams of UAVs could work together to map areas more quickly and, by measuring gradients of the distribution, quickly find large mosquito populations.

