\section[Hardware Design]{Hardware Design}
%%%%%%%%%%%%%%%%%%%%%%%%%%%%%%%%%%%%%%%

This section examines the components of the mosquito UAV system, shown in Fig.~\ref{fig:DroneAndNet}. This includes the UAV, electrified screen, surveying electronics, and a discussion of the energy budget. 
%The design for the mosquito UAV system been assigned a U.S.\ Provisional Patent Application~\cite{Becker2016patentapp}.

\subsection{The UAV}

The UAV is a custom-built, $\SI{177}{\centi\metre}$ wingspan hexacopter, controlled by a Pixhawk flight controller running ArduPilot Mega flight software. The UAV has a 3DR GPS module using the UBlox NEO-7 chipset.

\subsection{Screen Design}
The mosquito screen is designed to eliminate high density mosquito populations. 
This screen was constructed from two expanded aluminum mesh panels, spaced apart by \SI{3}{\milli\metre} thick ABS grid. 
These mesh panels have \SI{12}{\milli\metre} diamond-shaped  openings, and is held taught by nylon bolts around the perimeter.  
The bottom mesh panel is offset by half a diamond (6 mm) to the right to ensure all insects greater than 6 mm cannot pass through the net.
The top mesh is held at the reference voltage and the bottom mesh is energized to $1.8~kV$ above the reference voltage.

The perimeter is reinforced by two sets of \SI{7}{\milli\metre} diameter fiberglass rods that are inset into 3D printed corner fixtures.
These rods protect the frame from getting damaged from any side, and allows the UAV to land without damaging the net.


Once assembled, the net weighs \SI{0.948}{\kilogram} and has an overall area of \SI{0.194}{\square\metre}, with the spacer occupying \SI{0.0325}{\square\metre}. 
This makes the effective net area \SI{0.161}{\square\metre}. %, and the total cost of building a net this size is only \$27.44 USD.



%Design files and build instructions are available at~\cite{Vinh2016BugNet}.


%%%%%%%%%%%%%%%%%%%%%%%
\subsection{Screen Location}
The UAV carries the bug-zapping screen, which is suspended by paracord rope at each corner.  The location of this screen determines the efficacy of the mosquito UAV, measured in mosquitoes detected per second of flight time. The following describes a simplified analysis to optimize the screen location.

For manufacturing ease, the electrified screen is a rectangle with a width of $d_s$. The screen is suspended a distance $h_s$ beneath the UAV flying at height $h_d$.  We chose to suspend the screen beneath the UAV to avoid the weight of the rigid frame that would be required if the screen were above the UAV and because most mosquito species prefer low flight~\cite{gillies1976vertical}.  This screen can be suspended at any desired angle $\theta$ in comparison to horizontal, as shown in Fig.~\ref{fig:DroneConfigs}.
Two key parameters are the distance $h_s$ and the optimal angle $\theta$.  The goal is to clear the greatest volume of mosquitoes per second, a volume defined by the UAV forward velocity $v_f$ and the cross-sectional area $h_m \times d_s$ cleared by the screen, as shown in Fig.~\ref{fig:AngleVsSpeed}.

To hover, the UAV must push sufficient air down with velocity $v_d$ to apply a force that cancels the pull of gravity.  The UAV and screen combined have mass $m_{d}$ and its cross section can be approximated as a square with a side length of $d_d$.  The mass flow of air through the UAV's propellers is equal to the product of the change in velocity of the air, the density of the air $\rho_a$, and the cross sectional area.

\begin{figure}
\centering
\begin{overpic}[width=\columnwidth]{icra2018/DroneConfigs.pdf}\end{overpic}
\caption{\label{fig:DroneConfigs}
The UAV suspends a rectangular bug-zapping screen beneath it.  Propwash pushes incoming mosquitoes downwards, and the UAV clears a volume $h_m \times d_s \times v_f$ each second. Circles show two mosquitoes at equal time intervals relative to the UAV.} 
\vspace{-1em}
\end{figure}


We assume that air above the UAV is quiescent, so the change in velocity of the air is $v_d~ \si{\metre\per\second}$. So that 
\begin{align} \label{eq:forceBalanceForDrone}
\text{Force gravity} & = \left(\text{mass flow}\right) \cdot \text{air velocity and } \nonumber \\
m_{d} \cdot  g &= (v_d \cdot  \rho_a \cdot  d_d^2 ) \cdot  v_d. 
\end{align}

Then the required \emph{propwash}, the velocity of air beneath the UAV, for hovering is
\begin{align} \label{eq:dronePropwash}
v_d = \sqrt{ \frac{ m_d g}{\rho_a d_d^2} }.
\end{align}
The flight testing site in Houston, Texas is \SI{15}{\metre} above sea level. At sea level the density of air $\rho_a$ is \SI{1.225}{\kilogram\per\cubic\metre}.
The UAV and instrumentation combined weigh \SI{5.1}{\kilogram} with a width of \SI{0.75}{\metre}. The acceleration due to gravity is \SI{9.871}{\metre\per\square\second}.  Substituting these values gives $v_d = \SI{8.5}{\metre\per\second}$.

Due to propwash, an initially hovering mosquito will fall when under the UAV at a rate of $v_d$.  Relative to the UAV, the mosquito moves horizontally at a rate of $-v_f$.  As shown in Fig.~\ref{fig:DroneConfigs}, we can extend lines with slope $-v_d/v_f$ from the screen's trailing edge to $h_{\textrm{top}}$ and from the leading edge to $h_{\textrm{bottom}}$.
\begin{align} \label{eq:ClearedCrossSection}
h_{\textrm{top}} &= h_d - h_s + \frac{d_s}{2} \sin(\theta) +  \frac{d_d + d_s\cos(\theta)}{2}  \frac{v_d}{v_f} \nonumber \\
h_{\textrm{bottom}} &= h_d - h_s - \frac{d_s}{2} \sin(\theta) +  \frac{d_d - d_s\cos(\theta)}{2}  \frac{v_d}{v_f}  \nonumber \\
h_m &= h_{\textrm{top}} - h_{\textrm{bottom}} =  d_s\left(\frac{v_d}{v_f}\cos(\theta) + \sin(\theta) \right)
\end{align}
The optimal angle is therefore a function of forward and propwash velocity:
\begin{align} \label{eq:OptimalScreenAngle}
\ \theta = \mathrm{ArcTan}\left(\frac{v_f}{v_d}\right).
\end{align}

To ensure the maximum number of mosquitoes are collected, the screen must be sufficiently far below the UAV $ h_s > \frac{d_s}{2} \sin(\theta) +  \frac{d_d + d_s\cos(\theta)}{2}  \frac{v_d}{v_f}$  and the bottom of the screen must not touch the ground, $ h_d > h_s + \frac{d_s}{2} \sin(\theta) $.

\begin{figure}
\centering
\begin{overpic}[width=\columnwidth]{icra2018/AngleVsSpeed.pdf}\end{overpic}
\caption{\label{fig:AngleVsSpeed}
The volume cleared by a UAV is a function of screen angle $\theta$ and forward velocity $v_f$.  Dotted line shows the optimal angle given in \eqref{eq:OptimalScreenAngle}. } 
\vspace{-1em}
\end{figure}

There are practical limits to $h_s$ as well.  Tests with $h_s > \SI{2}{\metre}$ were abandoned because the long length caused the screen to act as a pendulum, introducing dynamics that made the system difficult to fly.

Changing the flying height $h_d$ of the UAV will target different mosquito populations because mosquitoes are not distributed uniformly vertically. 
Gillies and Wilkes demonstrated that different species of mosquitoes prefer to fly at different heights~\cite{gillies1976vertical}. 

%%%%%%%%%%%%%%%%%%%%%%%
\subsection{Wind Tunnel Verification of Net Angle}

This section describes experiments run in a wind tunnel to verify the simplified net angle analysis in the previous section. 
Smoke streaklines were used to visualize the flow of air as it passed by the UAV.
Due to space constraints in the wind tunnel, a free-flying phantom 4 was used instead of the hexacopter used for carrying the zapper. 
The wind tunnel was set to a \SI{3}{\metre\per\second} flow speed, and the UAV manually flown in approximately stable hovering.
The solo UAV is \num{0.3} $\times$ \num{0.3} $\times$ \SI{0.2}{\metre}.  The windtunnel has a \SI{1}{\metre} $\times$ \SI{1}{\metre} cross section. 
%??? brand name of the smoking apparatus?
As seen from Fig.~\ref{fig:WindTunnel}, the proposed screen position captures free flowing air and air entrained by the UAV propellers.
This test encouraged us to mount the net as close to the UAV as possible, so that air, and flying mosquitos, entrained by the propellers are pushed into the net.


\begin{figure}
\centering
\begin{overpic}[width=1.0\columnwidth]{icra2018/WindTunnelv01lowres.pdf}\end{overpic}
\caption{\label{fig:WindTunnel}
	Frames from wind tunnel test with free-flying UAV at \SI{3}{\metre\per\second} windspeed with smoke for streaklines~\cite{Bhatnagar2018}.  As shown in the frames at right, the proposed screen position (in red) captures free flowing air and air entrained by the UAV propellers.
	Each black square is \SI{25.4}{\milli\metre} in width.
} \vspace{-1em}
\end{figure}


%%%%%%%%%%%%%%%%%%%%%%%
\subsection{Data Logger}

The electrical detection and logging system is powered by a $9~V$ lithium ion battery applied directly to the controller and two AA $3~V$ lithium ion batteries applied to the power circuit for the screen.
The controller uses a GPS shield for monitoring the location and altitude as well as a real time clock to timestamp each data point collected from the system.
A Raspberry Pi 3 is used for data logging, 
sensors include a GPS sensor (NEO-6M Ublox), 
a capacitive humidity sensor, a thermistor (DHT22),
%ADS1115 16-bit ADC for monitoring the supply-side voltage of the net, and 
and an INA219 high side, 12-bit DC current sensor for monitoring the supply-side current delivered to the net.
The net current draw is logged at 100 Hz, while GPS and weather sensor data is logged at 1Hz.  
All data is stored on an onboard SD card.

%%%%%%%%%%%%%%%%%%%%%%%
\subsection{Energy Budget}

\begin{figure}
\centering
\begin{overpic}[width=0.9\columnwidth]{icra2018/OscilloscopeTrace.pdf}\end{overpic}
\caption{\label{fig:BugZapTrace}
					  Current, voltage, and power traces for five \textit{Culex quinquefasciatus} mosquitoes as each contacts the bug-zapping screen at $t=0$.  Contact causes a brief short that recovers in $\SI{160}{\milli\second}$.
} 
\vspace{-1em}
\end{figure}
% \todo{what is the new energy usage of the screen?}


Tests with an oscilloscope show that in the steady state, a $\SI{30.5}{\centi\metre} \times \SI{61}{\centi\metre}$ screen and electronics have a power consumption of \SI{3.6}{\watt}.  During a zap, the screen voltage monitoring circuit shorts briefly when the mosquito contacts the screen.  Figure~\ref{fig:BugZapTrace} shows the time sequences for battery and screen voltages, current, and power during five mosquito zaps.  
% % WE COULD NOT JUSTIFY THIS EQUATION 
%%  The initial current spike recovery can be modeled with an exponential fit.
%%
% \begin{align} \label{eq:BugZapFit}
%i=69.1e^{-2.7\times10^4 t} ~A
%\end{align}
%%
%The fit in \eqref{eq:BugZapFit} gives a time constant of $36~\mu s$ for the short and a recovery time of $80~\mu s$. 
Multiplying voltage by current to find the instantaneous power ($p=iv$) and integrating the area under the power curve show a total energy consumption of \SI{4.2}{\milli\joule} for each zap.  Recharging the screen requires more power and is represented in the latter part of the curves.  The overall recovery time is about $\SI{160}{\milli\second}$.  Most of the energy is consumed charging and maintaining the charge on the screen rather than in zapping the mosquitoes.





%Data
%The files used for this run of simulations are as follows:
%MosquitoFlightSimv2p0.m
%MosquitoSimv2p2randombounce.m
%MosquitoSimv2p2boustrophedon.m
%MosquitoSimComparev1p0.m
%
%All are located in the Dropbox code folder.
%Raw data are located in the Excel file Simulation Results.xlsx.
%    
