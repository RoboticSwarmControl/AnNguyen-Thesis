\section[Overview and Related Work]{Overview and Related Work}
%%%%%%%%%%%%%%%%%%%%%%%%%%%  

\noindent  \emph{Mosquito Control Solutions}:
Mosquito control also has a long history of efforts associated both with monitoring mosquito populations~\cite{dennett2007associations} and with eliminating mosquitoes.
The work involves both draining potential breeding grounds and destroying living mosquitoes~\cite{peter2005tick}.
An array of insecticidal compounds has been used with different application methods, concentrations, and quantities, including both larvicides and compounds directed at adult mosquitoes~\cite{larvicides2005guidelines}.

Various traps have been designed to capture and/or kill mosquitoes with increasing sophistication in imitating human bait, as designers strive to achieve a trap that can rival the attraction of a live human~\cite{maliti2015development}.
In recent history, methods have also included genetically modifying mosquitoes so that they either cannot reproduce effectively or cannot transmit diseases successfully~\cite{marshall2009malaria}, and with the recent genomic mapping of mosquito species, new ideas for more targeted work have been formulated~\cite{hill2005arthropod}.

Popular methods to control mosquitoes such as insecticides are effective, but they have the potential to introduce long-term environmental damage and mosquitoes have demonstrated the ability to become resistant to pesticides~\cite{ndiath2012resistance}.
Traditional electrified screens (bug zappers) use UV light to attract pests but have a large bycatch of non-pest insects~\cite{University-Of-Florida1997}
This chapter introduces techniques using bug zappers mounted on unmanned vehicles to autonomously seek out and eliminate mosquitoes in their breeding grounds and swarms.
Instrumentation on the bug zappers logs the GPS location, altitude, weather details, and time of each mosquito hit.
Mosquito control offices can use this information to analyze the insects' activities.
The device can be mounted on a remote-controlled or autonomous unmanned vehicle.
If autonomous, the vehicle can use the data collected from the electrified screen as feedback to improve the effectiveness of the motion plan. 
	
\noindent  \emph{Robotic Pest Management}:
As GPS technology has flourished and data processing has become cheaper and more readily available, researchers have explored options for implementing the new technologies in breeding ground removal~\cite{anupa2014identification} and more effective insecticide dispersion~\cite{hur2015low}.  Low-cost UAVs for residential spraying are under development~\cite{amenyo2014medizdroids}.  Even optical solutions have been considered, including laser containment~\cite{boonsri2012laser} or, by extension, exclusion and laser tracking and extermination~\cite{kare2010build}.
    
\noindent \emph{Robotic Coverage}: 
Robotic coverage has a long history. The basic problem is one of designing a path for a robot that ensures the robot visits within $r$ distance of every point on the workspace.  For an overview see~\cite{Choset2001}.  This work has been extended to use multiple coverage robots in a variety of ways, including using simple behaviors for the robots~\cite{spears2006physics,Koenig2001}.

The classic \emph{Travelling Salesman Problem (TSP)} is generalized to the \emph{Lawnmower Problem} \cite{arkin2000approximation}, which try to cover the most area with a tool of a nontrivial size.
For minimizing the turn cost of coverage path, Arkin et al.~\cite{arkin2005optimal} showed that finding minimum turn tours in grid graphs is NP-hard.
The complexity of finding a set of multiple cycles that cover a given set of locations at minimum total turn cost had remained elusive for many years; \emph{Problem~{53}} in \emph{The Open Problems Project}~\cite{openproblemproject} asks for the complexity of finding a minimum-cost (full) cycle cover in a 2-dimensional grid graph.
Arkin et al. showed~\cite{arkin2005optimal,arkin2001optimal} that the full coverage variant in {\em thin} grid graphs (which do not contain a $2\times 2$ square,
so every pixel is a boundary pixel) is solvable in polynomial time. Fekete et al.~\cite{dom3} were able to resolve this issue by showing that finding a cycle cover of minimum turn cost is  NP-hard.
