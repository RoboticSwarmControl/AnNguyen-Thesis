\chapter[UAV deploying Smartdarts]{UAV deployment of seismic microphones}

%SmartDarts, SmartSpiders and such.
%
%\input{introduction}
%%
%\section[Related Work]{Overview And Related Work}

This chapter presents a \emph{hetoerogenous seismic sensing system}, composed of stand alone geophone nodes deployed from unmanned aerial vehicles {UAV} and autonomous rovers with geophone attached.
We examine experimental data on geophone and soil coupling as a function of drop height and soil type.
We then provide a software tool for analyzing and planning a survey mission's logistic.
Our \emph{heterogenous sensor system} approach is designed to quickly and efficiently perform a survey with minimal manual labor for deployment and collection.

Sudarshan et al. \cite{sudarshan2015using} demonstrated a UAV equiped with four geophone sensors as landing gear.
The UAV in \cite{sudarshan2015using} can fly to a pre-programmed waypoint and land, attaching the geophones to the soil.

The geophones in  \cite{sudarshan2015using} had four problems:
(1) a UAV was required for each additional sensor,
(2) the force for planting the geophone was limited by the weight of the UAV,
(3) the platform required a level landing site,
(4) the magnets in the geophones distort compass readings, causing landing inaccuracy when autonomous.

The \emph{SeismicDart} presented in this system eliminates the need for a seperate UAV per sensor node.
Dropping the SeismicDarts from height also allows for greater penetration, firmer coupling and does not need a level landing site.
The new deployment unit also increase the distance between the SeismicDart's magnet and the UAV's magnetometer unit.

The SeismicSpider, our autonomous rover, can travel to survey locations inaccessible to UAV such as forests, thin atmosphere environments, caves or hard and rocky grounds which SeismicDarts cannot penetrate.
The SeismicSpider can also be deployed by a UAV, as close to the survey node as possible, then move to the desired location.

\subsection{Overview Of Seismic Sensing Theory}

\begin{figure}
	\centering
	\begin{overpic}[width=\columnwidth]{ral2016/Overview.pdf}\end{overpic}
	\caption{\label{fig:sensor_types}
	 Comparing state-of-the-art seismic survey sensors. a.) A traditional cabled system connects geophones in series to a seismic recorder and battery. b.) Autonomous nodal systems give each geophone a seismic recorder and battery.}
	%\vspace{-2em} 
\end{figure}



During seismic surveys, a source generates seismic waves that propagate under the earth's surface. 
These waves are sensed by geophone sensors and recorded for later analysis to detect the presence of resources. 
Fig.~\ref{fig:sensor_types} illustrates the components of current sensors. 

\subsubsection{Geophones}
Magnet-coil geophones contain a permanent magnet on a spring inside a coil. Voltage across the coil is proportional to velocity. 
 Beneath the coil housing is a metal spike. 
  Geophones are \emph{planted} by pushing this metal spike into the ground, which improves coupling with the ground to increase sensitivity. 
 The magnet-coil must be vertical. 
  Misalignment reduces the signal proportional to the cosine of the error.


\subsubsection{Cabled Systems}
Hydrocarbon exploration extensively uses traditional \emph{cabled systems} for seismic data acquisition.
Geophones are connected to each other in series using long cables. This cable is then connected to a seismic recorder and battery. 
The seismic recorder consists of a micro-controller which synchronizes the data acquired with a GPS signal and stores the data onboard. 
This method of data acquisition requires many manual laborers and a substantial expenditure for transporting the cables. 
Rugged terrain makes carrying and placing cables labor intensive, and the local manual labor pool may be unskilled or expensive.
   
\subsubsection{Autonomous Nodal Systems}
\emph{Autonomous nodal systems}~\cite{wood1998distributed} are now being used to conduct seismic surveys.
Unlike traditional cabled systems, autonomous nodal systems are not connected using cables.
The sensor, seismic recorder, and battery are all combined into a single package called a \emph{node} that can autonomously record data as shown in Fig.~\ref{fig:sensor_types}.
Even in these systems the data is generally stored in the onboard memory and can only be acquired after completing the survey.
This delay means errors cannot be detected and rectified while conducting the survey. 
Wireless autonomous nodes are a recent development.
These systems can transmit data wirelessly in real time~\cite{jiang2015geophysical}.
However, these systems still require manual laborers for planting the autonomous nodes at specific locations and deploying the large antennas necessary for wireless communication.
 
\subsection{Related Work}

Seismic surveying is a large industry.
The concept of using robots to place seismic sensors dates to the 1980s, when mobile robots placed seismic sensors on the moon~\cite{LSisMSE81}.
\cite{DSSMaA14} and \cite{coste2013seismic} proposed using a mobile robot for terrestrial geophone placement.
Plans are underway for a swarm of seismic sensors for Mars exploration~\cite{MAPL2006}.
Additionally,~\cite{muyzert2015marine} and~\cite{postel2014drone} proposed marine robots for hydrophone deployment underwater. 
Other work  focuses on data collection, using a UAV to wirelessly collect data from multiple sensors~\cite{wilcox2013seismic}.
Autonomous sensor deployment and mobile wireless sensor networks were studied in~\cite{howard2002mobile,corke2004autonomous,tuna2014autonomous}.
Heterogeneous mobile robotic teams were used for mapping and tracking in~\cite{howard2006experiments}.

\section[Related Work]{Overview And Related Work}

This chapter presents a \emph{hetoerogenous seismic sensing system}, composed of stand alone geophone nodes deployed from unmanned aerial vehicles {UAV} and autonomous rovers with geophone attached.
We examine experimental data on geophone and soil coupling as a function of drop height and soil type.
We then provide a software tool for analyzing and planning a survey mission's logistic.
Our \emph{heterogenous sensor system} approach is designed to quickly and efficiently perform a survey with minimal manual labor for deployment and collection.

Sudarshan et al. \cite{sudarshan2015using} demonstrated a UAV equiped with four geophone sensors as landing gear.
The UAV in \cite{sudarshan2015using} can fly to a pre-programmed waypoint and land, attaching the geophones to the soil.

The geophones in  \cite{sudarshan2015using} had four problems:
(1) a UAV was required for each additional sensor,
(2) the force for planting the geophone was limited by the weight of the UAV,
(3) the platform required a level landing site,
(4) the magnets in the geophones distort compass readings, causing landing inaccuracy when autonomous.

The \emph{SeismicDart} presented in this system eliminates the need for a seperate UAV per sensor node.
Dropping the SeismicDarts from height also allows for greater penetration, firmer coupling and does not need a level landing site.
The new deployment unit also increase the distance between the SeismicDart's magnet and the UAV's magnetometer unit.

The SeismicSpider, our autonomous rover, can travel to survey locations inaccessible to UAV such as forests, thin atmosphere environments, caves or hard and rocky grounds which SeismicDarts cannot penetrate.
The SeismicSpider can also be deployed by a UAV, as close to the survey node as possible, then move to the desired location.

\subsection{Overview Of Seismic Sensing Theory}

\begin{figure}
	\centering
	\begin{overpic}[width=\columnwidth]{ral2016/Overview.pdf}\end{overpic}
	\caption{\label{fig:sensor_types}
	 Comparing state-of-the-art seismic survey sensors. a.) A traditional cabled system connects geophones in series to a seismic recorder and battery. b.) Autonomous nodal systems give each geophone a seismic recorder and battery.}
	%\vspace{-2em} 
\end{figure}



During seismic surveys, a source generates seismic waves that propagate under the earth's surface. 
These waves are sensed by geophone sensors and recorded for later analysis to detect the presence of resources. 
Fig.~\ref{fig:sensor_types} illustrates the components of current sensors. 

\subsubsection{Geophones}
Magnet-coil geophones contain a permanent magnet on a spring inside a coil. Voltage across the coil is proportional to velocity. 
 Beneath the coil housing is a metal spike. 
  Geophones are \emph{planted} by pushing this metal spike into the ground, which improves coupling with the ground to increase sensitivity. 
 The magnet-coil must be vertical. 
  Misalignment reduces the signal proportional to the cosine of the error.


\subsubsection{Cabled Systems}
Hydrocarbon exploration extensively uses traditional \emph{cabled systems} for seismic data acquisition.
Geophones are connected to each other in series using long cables. This cable is then connected to a seismic recorder and battery. 
The seismic recorder consists of a micro-controller which synchronizes the data acquired with a GPS signal and stores the data onboard. 
This method of data acquisition requires many manual laborers and a substantial expenditure for transporting the cables. 
Rugged terrain makes carrying and placing cables labor intensive, and the local manual labor pool may be unskilled or expensive.
   
\subsubsection{Autonomous Nodal Systems}
\emph{Autonomous nodal systems}~\cite{wood1998distributed} are now being used to conduct seismic surveys.
Unlike traditional cabled systems, autonomous nodal systems are not connected using cables.
The sensor, seismic recorder, and battery are all combined into a single package called a \emph{node} that can autonomously record data as shown in Fig.~\ref{fig:sensor_types}.
Even in these systems the data is generally stored in the onboard memory and can only be acquired after completing the survey.
This delay means errors cannot be detected and rectified while conducting the survey. 
Wireless autonomous nodes are a recent development.
These systems can transmit data wirelessly in real time~\cite{jiang2015geophysical}.
However, these systems still require manual laborers for planting the autonomous nodes at specific locations and deploying the large antennas necessary for wireless communication.
 
\subsection{Related Work}

Seismic surveying is a large industry.
The concept of using robots to place seismic sensors dates to the 1980s, when mobile robots placed seismic sensors on the moon~\cite{LSisMSE81}.
\cite{DSSMaA14} and \cite{coste2013seismic} proposed using a mobile robot for terrestrial geophone placement.
Plans are underway for a swarm of seismic sensors for Mars exploration~\cite{MAPL2006}.
Additionally,~\cite{muyzert2015marine} and~\cite{postel2014drone} proposed marine robots for hydrophone deployment underwater. 
Other work  focuses on data collection, using a UAV to wirelessly collect data from multiple sensors~\cite{wilcox2013seismic}.
Autonomous sensor deployment and mobile wireless sensor networks were studied in~\cite{howard2002mobile,corke2004autonomous,tuna2014autonomous}.
Heterogeneous mobile robotic teams were used for mapping and tracking in~\cite{howard2006experiments}.

%%
%\input{smartdarts}
\input{chap/ral2016/smartdarts}
%%
%\input{seismicspider}
\input{chap/ral2016/seismicspider}
%%
%\input{uav}
\input{chap/ral2016/uav}
%%
%\section[Comparision]{Comparision}


\begin{figure} \centering
	{\includegraphics[width=\columnwidth]{ral2016/CannonPicture}}
	\caption{
		A pneumatic launcher for SeismicDarts.
		Ballistic dart deployment has limited usefulness because the incident angle is equal to the firing angle.} 
	\label{fig:CannonPicture}
\end{figure}

\begin{figure*}[htb]
	\centering 
	%\vspace{1em}
	%\renewcommand{\figwid}{0.48\columnwidth}
	\begin{overpic}[width =\figwid]{ral2016/sim1_1.pdf}
	\end{overpic}
	\begin{overpic}[width =\figwid]{ral2016/sim1_2.pdf}
	\end{overpic}
	\begin{overpic}[width =\figwid]{ral2016/sim1_3.pdf}
	\end{overpic}
	\begin{overpic}[width =\figwid]{ral2016/sim1_4.pdf}
	\end{overpic}
	\caption{
		Screenshots of simulations that were performed to estimate time take by different sensors surveying 100x100 m grid:
		a.) only SeismicSpiders
		b.) SeismicDarts and deployment system
		c.) heterogeneous system
		d.) human workers.
	\label{fig:Sim_overview}}
\end{figure*}

\subsection{Ballistic Deployment}
To compare an alternative deployment mechanism we built the pneumatic cannon shown in Fig.~\ref{fig:CannonPicture}a.
The pneumatic cannon is U-shaped,  2 m in length, with a 0.1 m (4 inch) diameter pressure chamber and a 0.08 m (3 inch) diameter firing barrel, connected by an electronic valve (Rain Bird JTV/ASF 100).
The cannon is aimed by selecting an appropriate firing angle $\theta_f$, azimuth angle, and chamber pressure.
The reachable workspace is an annular ring whose radius $r$ is a function of the firing angle and initial velocity $v$.
Neglecting air resistance, this range is found by integration:
\begin{align}
r = \frac{v^2}{g} \sin( 2 \theta_f ).
\end{align} 
Initial velocity is limited by the maximum pressure and size of the pressure chamber.
The cannon used  SCH 40 PVC, which is limited to a maximum pressure of 3 Mpa (450 psi).

We charged our system to 1 Mpa (150 psi), and achieved a range of $\approx$ 150 m.
This range is considerably smaller than the UAV's range, which when loaded can complete a round trip of $\approx 1.5$ km.

A larger problem, illustrated in Fig.~\ref{fig:CannonPicture}, is that angle of incidence $\theta_i$ is equal to the firing angle $\theta_f$.
Maximum range is achieved with $\theta_f = 45^\circ$, but this angle of incidence reduces the geophone sensitivity to $\cos(\theta_f )\approx 0.7$.
The placement accuracy of the cannon is lower than the UAV because a fired dart must fly over a longer distance than a dropped dart.
Safety reasons also limit applications for a pneumatic launcher.

\begin{table} \centering
	{\includegraphics[width=\columnwidth]{ral2016/simulation_table.pdf}}
	\caption{Comparison of different  deployment modes highlights the efficiency of UAV deployment.} 
	\label{tab:Sim_table}
\end{table}

\begin{figure} \centering
	{\includegraphics[width=\columnwidth]{ral2016/DronevsTime.pdf}}
	\caption{Survey time for a 1km x 10 km region for different numbers of UAVs.} 
	\label{fig:DronevsTime}
	%\vspace{-1em}
\end{figure}

\begin{figure} \centering
	{\includegraphics[width=\columnwidth]{ral2016/het_sen_ratio.pdf}}
	\caption{
		Survey time for different sensor ratios.
		The total number of sensors \{5000, 3000\} were kept constant.
		Ten darts were provided for each UAV.} 
	\label{fig:het_sen_ratio}
	%\vspace{-1em}
\end{figure}

\subsection{Simulation Studies}
A scheduling system to compare  time and costs for seismic surveys with varying numbers of UAVs, SeismicSpiders, SeismicDarts, and human laborers was coded in  {\sc Matlab}, available at \cite{Srikanth2016seismicScheduler}.
In each simulation, a seismic source must be measured at every survey point.
The scheduler must assign each sensor (SeismicDart or SeismicSpider) to an unmeasured survey point and assign each UAV or human worker a dart to pickup or deploy.
Once a sensor reaches a survey point, that sensor must wait until a seismic source is measured.
A vibration truck (blue) provides the seismic source.
Motion planning uses a centralized, greedy strategy.

Frames from four different cases on a small survey region are shown in Fig.~\ref{fig:Sim_overview} on a 100x100 m area with survey points at 10 m spacing:
a.) simulates 10 SeismicSpiders;  
b.) simulates 5 SeismicUAVs deploying 50 SeismicDarts.
Each UAV was allowed to carry up to 4 darts;
c.) simulates 10 SeismicSpider, 5 SeismicUAVs and 50 SeismicDarts;
d.) simulates 5 human workers deploying 75 SeismicDarts.
Each worker was allowed to carry up to 10 darts.
Survey points are grey circles if unmeasured and green if measured.
The simulation uses red hexagons for SeismicSpiders,  black diamonds for UAVs,  inverted yellow triangles for SeismicDarts, and magenta diamonds for human workers.
The assigned motion path for each sensor is colored magenta and the path completed is blue.

This tool allows us to examine engineering and logistic trade-offs quickly through simulations.
For example, Fig.~\ref{fig:DronevsTime} assumes a fixed number of darts and examines the finishing time with $5$ to $500$ UAVs.
The time required decays asymptotically, but $140$ UAVs requires only twice the amount of time required for $500$ UAVs, indicating  $140$ UAVs are sufficient for the task.   
 Substantial cost savings can be obtained by selecting the number of UAVs required to complete within a certain percentage greater than the optimal time.

The tool is useful for comparing the effectiveness of heterogeneous teams.
Table~\ref{tab:Sim_table} compares surveying a $1$ km x $10$ km strip of land with teams of (a) $5000$ SeismicSpiders, (b) $500$ UAVs and $5000$ SeismicDarts, (c) $500$ humans and $5000$ geophones.
Team (b) completed six times faster than team (c).
In Fig.~\ref{fig:het_sen_ratio}, the total number of mobile agents are constant, but the percentage of UAVs and SeismicSpiders are varied.
10 SeismicDarts were provided for each UAV.
Increasing the percentage of UAVs lowers the deployment time because UAVs move 20 m/s but SeismicSpiders move 0.2 m/s.
The velcity difference makes UAV deployment time-efficient.

\section[Comparision]{Comparision}


\begin{figure} \centering
	{\includegraphics[width=\columnwidth]{ral2016/CannonPicture}}
	\caption{
		A pneumatic launcher for SeismicDarts.
		Ballistic dart deployment has limited usefulness because the incident angle is equal to the firing angle.} 
	\label{fig:CannonPicture}
\end{figure}

\begin{figure*}[htb]
	\centering 
	%\vspace{1em}
	%\renewcommand{\figwid}{0.48\columnwidth}
	\begin{overpic}[width =\figwid]{ral2016/sim1_1.pdf}
	\end{overpic}
	\begin{overpic}[width =\figwid]{ral2016/sim1_2.pdf}
	\end{overpic}
	\begin{overpic}[width =\figwid]{ral2016/sim1_3.pdf}
	\end{overpic}
	\begin{overpic}[width =\figwid]{ral2016/sim1_4.pdf}
	\end{overpic}
	\caption{
		Screenshots of simulations that were performed to estimate time take by different sensors surveying 100x100 m grid:
		a.) only SeismicSpiders
		b.) SeismicDarts and deployment system
		c.) heterogeneous system
		d.) human workers.
	\label{fig:Sim_overview}}
\end{figure*}

\subsection{Ballistic Deployment}
To compare an alternative deployment mechanism we built the pneumatic cannon shown in Fig.~\ref{fig:CannonPicture}a.
The pneumatic cannon is U-shaped,  2 m in length, with a 0.1 m (4 inch) diameter pressure chamber and a 0.08 m (3 inch) diameter firing barrel, connected by an electronic valve (Rain Bird JTV/ASF 100).
The cannon is aimed by selecting an appropriate firing angle $\theta_f$, azimuth angle, and chamber pressure.
The reachable workspace is an annular ring whose radius $r$ is a function of the firing angle and initial velocity $v$.
Neglecting air resistance, this range is found by integration:
\begin{align}
r = \frac{v^2}{g} \sin( 2 \theta_f ).
\end{align} 
Initial velocity is limited by the maximum pressure and size of the pressure chamber.
The cannon used  SCH 40 PVC, which is limited to a maximum pressure of 3 Mpa (450 psi).

We charged our system to 1 Mpa (150 psi), and achieved a range of $\approx$ 150 m.
This range is considerably smaller than the UAV's range, which when loaded can complete a round trip of $\approx 1.5$ km.

A larger problem, illustrated in Fig.~\ref{fig:CannonPicture}, is that angle of incidence $\theta_i$ is equal to the firing angle $\theta_f$.
Maximum range is achieved with $\theta_f = 45^\circ$, but this angle of incidence reduces the geophone sensitivity to $\cos(\theta_f )\approx 0.7$.
The placement accuracy of the cannon is lower than the UAV because a fired dart must fly over a longer distance than a dropped dart.
Safety reasons also limit applications for a pneumatic launcher.

\begin{table} \centering
	{\includegraphics[width=\columnwidth]{ral2016/simulation_table.pdf}}
	\caption{Comparison of different  deployment modes highlights the efficiency of UAV deployment.} 
	\label{tab:Sim_table}
\end{table}

\begin{figure} \centering
	{\includegraphics[width=\columnwidth]{ral2016/DronevsTime.pdf}}
	\caption{Survey time for a 1km x 10 km region for different numbers of UAVs.} 
	\label{fig:DronevsTime}
	%\vspace{-1em}
\end{figure}

\begin{figure} \centering
	{\includegraphics[width=\columnwidth]{ral2016/het_sen_ratio.pdf}}
	\caption{
		Survey time for different sensor ratios.
		The total number of sensors \{5000, 3000\} were kept constant.
		Ten darts were provided for each UAV.} 
	\label{fig:het_sen_ratio}
	%\vspace{-1em}
\end{figure}

\subsection{Simulation Studies}
A scheduling system to compare  time and costs for seismic surveys with varying numbers of UAVs, SeismicSpiders, SeismicDarts, and human laborers was coded in  {\sc Matlab}, available at \cite{Srikanth2016seismicScheduler}.
In each simulation, a seismic source must be measured at every survey point.
The scheduler must assign each sensor (SeismicDart or SeismicSpider) to an unmeasured survey point and assign each UAV or human worker a dart to pickup or deploy.
Once a sensor reaches a survey point, that sensor must wait until a seismic source is measured.
A vibration truck (blue) provides the seismic source.
Motion planning uses a centralized, greedy strategy.

Frames from four different cases on a small survey region are shown in Fig.~\ref{fig:Sim_overview} on a 100x100 m area with survey points at 10 m spacing:
a.) simulates 10 SeismicSpiders;  
b.) simulates 5 SeismicUAVs deploying 50 SeismicDarts.
Each UAV was allowed to carry up to 4 darts;
c.) simulates 10 SeismicSpider, 5 SeismicUAVs and 50 SeismicDarts;
d.) simulates 5 human workers deploying 75 SeismicDarts.
Each worker was allowed to carry up to 10 darts.
Survey points are grey circles if unmeasured and green if measured.
The simulation uses red hexagons for SeismicSpiders,  black diamonds for UAVs,  inverted yellow triangles for SeismicDarts, and magenta diamonds for human workers.
The assigned motion path for each sensor is colored magenta and the path completed is blue.

This tool allows us to examine engineering and logistic trade-offs quickly through simulations.
For example, Fig.~\ref{fig:DronevsTime} assumes a fixed number of darts and examines the finishing time with $5$ to $500$ UAVs.
The time required decays asymptotically, but $140$ UAVs requires only twice the amount of time required for $500$ UAVs, indicating  $140$ UAVs are sufficient for the task.   
 Substantial cost savings can be obtained by selecting the number of UAVs required to complete within a certain percentage greater than the optimal time.

The tool is useful for comparing the effectiveness of heterogeneous teams.
Table~\ref{tab:Sim_table} compares surveying a $1$ km x $10$ km strip of land with teams of (a) $5000$ SeismicSpiders, (b) $500$ UAVs and $5000$ SeismicDarts, (c) $500$ humans and $5000$ geophones.
Team (b) completed six times faster than team (c).
In Fig.~\ref{fig:het_sen_ratio}, the total number of mobile agents are constant, but the percentage of UAVs and SeismicSpiders are varied.
10 SeismicDarts were provided for each UAV.
Increasing the percentage of UAVs lowers the deployment time because UAVs move 20 m/s but SeismicSpiders move 0.2 m/s.
The velcity difference makes UAV deployment time-efficient.

%%
%\section{Conclusion and Future Work}\label{sec:conclusion}

This chapter presented an approach for finding optimal tours given turn costs and an energy budget, inspired by a mosquito-killing UAV with limited battery life. 
Initial experiments with the UAV and electrified screen track the location of a mosquito-killing UAV as it patrols a field and maps mosquito kills.  

%Future work
Many refinements to the algorithm could be pursued in future work, including changes to both the mosquito-biasing algorithm and the robot flight simulation.  The model may be expanded to continuous space, three dimensions, and to arbitrary turn angles.  These and other considerations will make a more realistic model for future work.  

Further testing of the multi-copter UAV is indicated and will allow more extensive testing of the robustness and accuracy of the hardware design. New sensors that can identify and detect flying insects~\cite{chen2014flying} may be added to the UAV and enable it to proactively steer toward insect swarms and identify insects in realtime.

The concept may be extended to a non-destructive population survey in which the screen could be replaced with a net and, with appropriate lighting, the camera used to record capture events.  Teams of UAVs could work together to map areas more quickly and, by measuring gradients of the distribution, quickly find large mosquito populations.


\section[Conclusion]{Future Work}

This chapter presented a \emph{heterogeneous sensor system} and technique for autonomous geophone deployment.
The \emph{heterogeneous sensor system} compose of two components, UAV deployable SeismicDarts, mobile SeismicSpider.
The work in this chapter allow us to automate tasks that currently require a much more manual labors in hazardous environment.

The SeismicDart's output is comparable to well-planted geophones. 
For hard surfaces where the SeismicDart could not penetrate, we presented an autonomous alternative, the SeismicSpider.  
The SeismicSpider is mobile, can actively adjust its sensors to ensure ground contact and vertical placement, and can be deployed and retrieved by UAVs.

Autonomous deployment was conducted using GPS, proving human involvement could be minimized by adopting the proposed technique.
Hardware experiments compared the autonomous system to manual planting and ballistic deployment.
Simulation studies show time and cost savings over traditional manual techniques.

Future systems should be weatherized and optimized for cost, robustness, range, and speed.
Soil maps could be used to plan a survey, allocating SeismicSpiders to rocky or forested areas and SeismicDarts to penetrable soils.
These maps can be made more accurate using drone-carried ground penetrating radar \cite{merz2015new}.
Alternatively, the  SeismicDart's internal accelerometer also provides feedback on the quality of the plant.
As shown in Fig.~\ref{fig:AnglePlotIndoors}, angular deviations indicate a higher drop height is needed.

%
%
%\title{A Heterogeneous Robotics Team for Large-Scale Seismic Sensing} 
%%  compress using: gs -sDEVICE=pdfwrite -dCompatibilityLevel=1.4 -dNOPAUSE -dQUIET -dBATCH      -sOutputFile=RA-L2016compressed.pdf RA-L2016.pdf
%\documentclass[letterpaper, 10 pt, journal, twoside]{IEEEtran}
%\IEEEoverridecommandlockouts                              % This command is only needed if 
%                                                          % you want to use the \thanks command
%
%%\overrideIEEEmargins                                      % Needed to meet printer requirements.
%%\IEEEoverridecommandlockout
%%\documentclass[conference]{IEEEtran}
%\newcommand{\subparagraph}{}
%\usepackage{epsfig,graphicx,cite}
%\usepackage{psfrag}
%%\usepackage[small,compact]{titlesec}
%\usepackage{wrapfig}
%\usepackage{mathrsfs}
%\usepackage{bm}
%\usepackage{cite,url,subfigure,epsfig,graphicx}
%\usepackage{verbatim,amsfonts,amsmath,amssymb}
%\usepackage{fancyhdr}
%\usepackage{mathbbold}
%\usepackage{bbm}
%\usepackage{mathrsfs}
%\usepackage{amsfonts}
%\usepackage{cite,url,subfigure,epsfig,graphicx}
%\usepackage{amssymb,amsmath,bm,makecell}
%\usepackage{indentfirst}
%\usepackage{overpic}
%\newcommand{\figwid}{0.22\columnwidth}
%
%\usepackage{amsmath}
%\usepackage{algorithm}
%\usepackage[noend]{algpseudocode}
%
%\usepackage[T1]{fontenc}
%\usepackage[utf8]{inputenc}
%\usepackage{authblk}
%
%
%
%\usepackage{mathtools}
%\usepackage[font=footnotesize]{caption}
%\usepackage{amsmath}
%\usepackage{amssymb}
%\usepackage{tabulary}
%\usepackage{booktabs}
%\usepackage{framed}
%\usepackage{fancyhdr}
%%\usepackage[hypertex]{hyperref}
%\usepackage[hidelinks]{hyperref}
%%\IEEEoverridecommandlockouts
%\usepackage{cite,url,subfigure,epsfig,graphicx}
%\usepackage{times,verbatim,amsfonts,amsmath,color}
%%\newtheorem{definition}{\textbf{Definition}}
%%\newtheorem{lemma}{\textbf{Lemma}}
%%\newtheorem{proof}{\textbf{Proof}}
%%\newtheorem{theorem}{\textbf{Theorem}}
%%\newtheorem{example}{\textbf{Example}}
%%\newtheorem{proposition}{\textbf{Proposition}}
%%\newtheorem{remark}{\textbf{Remark}}
%%\newtheorem{corrolary}{\textbf{Corrolary}}
%%\newtheorem{ex}{\textbf{EX}}
%\usepackage{overpic}
%\graphicspath{{./},{./pictures/}}
%\setcounter{secnumdepth}{4}
%\setcounter{tocdepth}{4}
%\usepackage[table,xcdraw]{xcolor}
%\newcommand{\todo}[1]{ \textcolor{red}{\ttfamily#1}}
%\newcommand{\todobox}[1]{\vspace{5 mm}\par \noindent \framebox{\begin{minipage}[c]{0.98 \columnwidth} \ttfamily\flushleft \textcolor{red}{#1}\end{minipage}}\vspace{5 mm}\par}
%\let\labelindent\relax \usepackage{enumitem}
%
%
%
%
%
%\begin{document}
%%
%% paper title
%% can use linebreaks \\ within to get better formatting as desired
%\title{A Heterogeneous Robotics Team for Large-Scale Seismic Sensing} 
%% Paper headers
%\markboth{IEEE Robotics and Automation Letters. Preprint Version. Accepted Jan, 2017}
%{Sudarshan \MakeLowercase{\textit{et al.}}: Robotics Team for Seismic Sensing}
%% Use only for final RAL version
%
%\author{Srikanth K. V. Sudarshan$^{1}$,
%Victor Montano$^{1}$,
%An Nguyen$^{1}$,\\ 
%Michael McClimans$^{2}$,
%Li Chang$^{2}$,
%Robert R. Stewart$^{2}$, and
% Aaron T. Becker$^{1}$% <-this % stops a space
% \thanks{Manuscript received: September 10, 2016; Revised November 7, 2016; Accepted January 22, 2017.}%Use only for final RALversion
%\thanks{This paper was recommended for publication by Editor Roberts, Jonathan upon evaluation of the Associate Editor and Reviewers' comments. *This work was supported by the National Science Foundation under Grant No.\ \href{http://nsf.gov/awardsearch/showAward?AWD_ID=1553063}{ [IIS-1553063]}.}% <-this % stops a space
%\thanks{$^{1}$Department of Electrical and Computer Engineering,}
%\thanks{$^{2}$Department of Earth and Atmospheric Sciences,\\ University of Houston, 4800 Calhoun Rd, Houston, TX 77004, USA
%        {\tt\small \{skvenkatasudarshan, vjmontano, anguyen43, msmcclimans, lchang13, rrstewart, atbecker\}@uh.edu}}%
%\thanks{Digital Object Identifier (DOI): see top of this page.}
%}
%
%
%
%\maketitle
%%\thispagestyle{empty}
%%\pagestyle{empty}
%
%
%\begin{abstract} 
%Seismic surveying requires placing a large number of sensors (geophones) in a grid pattern, triggering a seismic event, and recording vibration readings. 
% The goal of the surveying is often to locate subsurface resources.  
%Traditional seismic surveying employs human laborers for sensor placement and retrieval. 
%The major drawbacks of surveying with human deployment are the high costs and time, and risks to humans due to explosives, terrain, and climatic conditions.
%We propose an autonomous, heterogeneous sensor deployment system using unmanned aerial vehicles (UAVs) to deploy mobile and immobile sensors.
%The proposed system begins to overcome some of the problems associated with traditional systems.  
%This paper provides detailed analysis and comparison with traditional survey techniques. 
%Hardware experiments and simulations show promise for automation reducing cost and time. 
% Autonomous aerial systems will have a substantial contribution to make in future seismic surveys. 
%\end{abstract}
%
%\begin{IEEEkeywords}
%	Aerial Robotics; Robotics in Hazardous Fields; Distributed Robot Systems
%\end{IEEEkeywords}
%
%\input{introduction}
%%
%\section[Related Work]{Overview And Related Work}

This chapter presents a \emph{hetoerogenous seismic sensing system}, composed of stand alone geophone nodes deployed from unmanned aerial vehicles {UAV} and autonomous rovers with geophone attached.
We examine experimental data on geophone and soil coupling as a function of drop height and soil type.
We then provide a software tool for analyzing and planning a survey mission's logistic.
Our \emph{heterogenous sensor system} approach is designed to quickly and efficiently perform a survey with minimal manual labor for deployment and collection.

Sudarshan et al. \cite{sudarshan2015using} demonstrated a UAV equiped with four geophone sensors as landing gear.
The UAV in \cite{sudarshan2015using} can fly to a pre-programmed waypoint and land, attaching the geophones to the soil.

The geophones in  \cite{sudarshan2015using} had four problems:
(1) a UAV was required for each additional sensor,
(2) the force for planting the geophone was limited by the weight of the UAV,
(3) the platform required a level landing site,
(4) the magnets in the geophones distort compass readings, causing landing inaccuracy when autonomous.

The \emph{SeismicDart} presented in this system eliminates the need for a seperate UAV per sensor node.
Dropping the SeismicDarts from height also allows for greater penetration, firmer coupling and does not need a level landing site.
The new deployment unit also increase the distance between the SeismicDart's magnet and the UAV's magnetometer unit.

The SeismicSpider, our autonomous rover, can travel to survey locations inaccessible to UAV such as forests, thin atmosphere environments, caves or hard and rocky grounds which SeismicDarts cannot penetrate.
The SeismicSpider can also be deployed by a UAV, as close to the survey node as possible, then move to the desired location.

\subsection{Overview Of Seismic Sensing Theory}

\begin{figure}
	\centering
	\begin{overpic}[width=\columnwidth]{ral2016/Overview.pdf}\end{overpic}
	\caption{\label{fig:sensor_types}
	 Comparing state-of-the-art seismic survey sensors. a.) A traditional cabled system connects geophones in series to a seismic recorder and battery. b.) Autonomous nodal systems give each geophone a seismic recorder and battery.}
	%\vspace{-2em} 
\end{figure}



During seismic surveys, a source generates seismic waves that propagate under the earth's surface. 
These waves are sensed by geophone sensors and recorded for later analysis to detect the presence of resources. 
Fig.~\ref{fig:sensor_types} illustrates the components of current sensors. 

\subsubsection{Geophones}
Magnet-coil geophones contain a permanent magnet on a spring inside a coil. Voltage across the coil is proportional to velocity. 
 Beneath the coil housing is a metal spike. 
  Geophones are \emph{planted} by pushing this metal spike into the ground, which improves coupling with the ground to increase sensitivity. 
 The magnet-coil must be vertical. 
  Misalignment reduces the signal proportional to the cosine of the error.


\subsubsection{Cabled Systems}
Hydrocarbon exploration extensively uses traditional \emph{cabled systems} for seismic data acquisition.
Geophones are connected to each other in series using long cables. This cable is then connected to a seismic recorder and battery. 
The seismic recorder consists of a micro-controller which synchronizes the data acquired with a GPS signal and stores the data onboard. 
This method of data acquisition requires many manual laborers and a substantial expenditure for transporting the cables. 
Rugged terrain makes carrying and placing cables labor intensive, and the local manual labor pool may be unskilled or expensive.
   
\subsubsection{Autonomous Nodal Systems}
\emph{Autonomous nodal systems}~\cite{wood1998distributed} are now being used to conduct seismic surveys.
Unlike traditional cabled systems, autonomous nodal systems are not connected using cables.
The sensor, seismic recorder, and battery are all combined into a single package called a \emph{node} that can autonomously record data as shown in Fig.~\ref{fig:sensor_types}.
Even in these systems the data is generally stored in the onboard memory and can only be acquired after completing the survey.
This delay means errors cannot be detected and rectified while conducting the survey. 
Wireless autonomous nodes are a recent development.
These systems can transmit data wirelessly in real time~\cite{jiang2015geophysical}.
However, these systems still require manual laborers for planting the autonomous nodes at specific locations and deploying the large antennas necessary for wireless communication.
 
\subsection{Related Work}

Seismic surveying is a large industry.
The concept of using robots to place seismic sensors dates to the 1980s, when mobile robots placed seismic sensors on the moon~\cite{LSisMSE81}.
\cite{DSSMaA14} and \cite{coste2013seismic} proposed using a mobile robot for terrestrial geophone placement.
Plans are underway for a swarm of seismic sensors for Mars exploration~\cite{MAPL2006}.
Additionally,~\cite{muyzert2015marine} and~\cite{postel2014drone} proposed marine robots for hydrophone deployment underwater. 
Other work  focuses on data collection, using a UAV to wirelessly collect data from multiple sensors~\cite{wilcox2013seismic}.
Autonomous sensor deployment and mobile wireless sensor networks were studied in~\cite{howard2002mobile,corke2004autonomous,tuna2014autonomous}.
Heterogeneous mobile robotic teams were used for mapping and tracking in~\cite{howard2006experiments}.

%%
%\input{smartdarts}
%%
%\input{seismicspider}
%%
%\input{uav}
%%
%\section[Comparision]{Comparision}


\begin{figure} \centering
	{\includegraphics[width=\columnwidth]{ral2016/CannonPicture}}
	\caption{
		A pneumatic launcher for SeismicDarts.
		Ballistic dart deployment has limited usefulness because the incident angle is equal to the firing angle.} 
	\label{fig:CannonPicture}
\end{figure}

\begin{figure*}[htb]
	\centering 
	%\vspace{1em}
	%\renewcommand{\figwid}{0.48\columnwidth}
	\begin{overpic}[width =\figwid]{ral2016/sim1_1.pdf}
	\end{overpic}
	\begin{overpic}[width =\figwid]{ral2016/sim1_2.pdf}
	\end{overpic}
	\begin{overpic}[width =\figwid]{ral2016/sim1_3.pdf}
	\end{overpic}
	\begin{overpic}[width =\figwid]{ral2016/sim1_4.pdf}
	\end{overpic}
	\caption{
		Screenshots of simulations that were performed to estimate time take by different sensors surveying 100x100 m grid:
		a.) only SeismicSpiders
		b.) SeismicDarts and deployment system
		c.) heterogeneous system
		d.) human workers.
	\label{fig:Sim_overview}}
\end{figure*}

\subsection{Ballistic Deployment}
To compare an alternative deployment mechanism we built the pneumatic cannon shown in Fig.~\ref{fig:CannonPicture}a.
The pneumatic cannon is U-shaped,  2 m in length, with a 0.1 m (4 inch) diameter pressure chamber and a 0.08 m (3 inch) diameter firing barrel, connected by an electronic valve (Rain Bird JTV/ASF 100).
The cannon is aimed by selecting an appropriate firing angle $\theta_f$, azimuth angle, and chamber pressure.
The reachable workspace is an annular ring whose radius $r$ is a function of the firing angle and initial velocity $v$.
Neglecting air resistance, this range is found by integration:
\begin{align}
r = \frac{v^2}{g} \sin( 2 \theta_f ).
\end{align} 
Initial velocity is limited by the maximum pressure and size of the pressure chamber.
The cannon used  SCH 40 PVC, which is limited to a maximum pressure of 3 Mpa (450 psi).

We charged our system to 1 Mpa (150 psi), and achieved a range of $\approx$ 150 m.
This range is considerably smaller than the UAV's range, which when loaded can complete a round trip of $\approx 1.5$ km.

A larger problem, illustrated in Fig.~\ref{fig:CannonPicture}, is that angle of incidence $\theta_i$ is equal to the firing angle $\theta_f$.
Maximum range is achieved with $\theta_f = 45^\circ$, but this angle of incidence reduces the geophone sensitivity to $\cos(\theta_f )\approx 0.7$.
The placement accuracy of the cannon is lower than the UAV because a fired dart must fly over a longer distance than a dropped dart.
Safety reasons also limit applications for a pneumatic launcher.

\begin{table} \centering
	{\includegraphics[width=\columnwidth]{ral2016/simulation_table.pdf}}
	\caption{Comparison of different  deployment modes highlights the efficiency of UAV deployment.} 
	\label{tab:Sim_table}
\end{table}

\begin{figure} \centering
	{\includegraphics[width=\columnwidth]{ral2016/DronevsTime.pdf}}
	\caption{Survey time for a 1km x 10 km region for different numbers of UAVs.} 
	\label{fig:DronevsTime}
	%\vspace{-1em}
\end{figure}

\begin{figure} \centering
	{\includegraphics[width=\columnwidth]{ral2016/het_sen_ratio.pdf}}
	\caption{
		Survey time for different sensor ratios.
		The total number of sensors \{5000, 3000\} were kept constant.
		Ten darts were provided for each UAV.} 
	\label{fig:het_sen_ratio}
	%\vspace{-1em}
\end{figure}

\subsection{Simulation Studies}
A scheduling system to compare  time and costs for seismic surveys with varying numbers of UAVs, SeismicSpiders, SeismicDarts, and human laborers was coded in  {\sc Matlab}, available at \cite{Srikanth2016seismicScheduler}.
In each simulation, a seismic source must be measured at every survey point.
The scheduler must assign each sensor (SeismicDart or SeismicSpider) to an unmeasured survey point and assign each UAV or human worker a dart to pickup or deploy.
Once a sensor reaches a survey point, that sensor must wait until a seismic source is measured.
A vibration truck (blue) provides the seismic source.
Motion planning uses a centralized, greedy strategy.

Frames from four different cases on a small survey region are shown in Fig.~\ref{fig:Sim_overview} on a 100x100 m area with survey points at 10 m spacing:
a.) simulates 10 SeismicSpiders;  
b.) simulates 5 SeismicUAVs deploying 50 SeismicDarts.
Each UAV was allowed to carry up to 4 darts;
c.) simulates 10 SeismicSpider, 5 SeismicUAVs and 50 SeismicDarts;
d.) simulates 5 human workers deploying 75 SeismicDarts.
Each worker was allowed to carry up to 10 darts.
Survey points are grey circles if unmeasured and green if measured.
The simulation uses red hexagons for SeismicSpiders,  black diamonds for UAVs,  inverted yellow triangles for SeismicDarts, and magenta diamonds for human workers.
The assigned motion path for each sensor is colored magenta and the path completed is blue.

This tool allows us to examine engineering and logistic trade-offs quickly through simulations.
For example, Fig.~\ref{fig:DronevsTime} assumes a fixed number of darts and examines the finishing time with $5$ to $500$ UAVs.
The time required decays asymptotically, but $140$ UAVs requires only twice the amount of time required for $500$ UAVs, indicating  $140$ UAVs are sufficient for the task.   
 Substantial cost savings can be obtained by selecting the number of UAVs required to complete within a certain percentage greater than the optimal time.

The tool is useful for comparing the effectiveness of heterogeneous teams.
Table~\ref{tab:Sim_table} compares surveying a $1$ km x $10$ km strip of land with teams of (a) $5000$ SeismicSpiders, (b) $500$ UAVs and $5000$ SeismicDarts, (c) $500$ humans and $5000$ geophones.
Team (b) completed six times faster than team (c).
In Fig.~\ref{fig:het_sen_ratio}, the total number of mobile agents are constant, but the percentage of UAVs and SeismicSpiders are varied.
10 SeismicDarts were provided for each UAV.
Increasing the percentage of UAVs lowers the deployment time because UAVs move 20 m/s but SeismicSpiders move 0.2 m/s.
The velcity difference makes UAV deployment time-efficient.

%%
%\section{Conclusion and Future Work}\label{sec:conclusion}

This chapter presented an approach for finding optimal tours given turn costs and an energy budget, inspired by a mosquito-killing UAV with limited battery life. 
Initial experiments with the UAV and electrified screen track the location of a mosquito-killing UAV as it patrols a field and maps mosquito kills.  

%Future work
Many refinements to the algorithm could be pursued in future work, including changes to both the mosquito-biasing algorithm and the robot flight simulation.  The model may be expanded to continuous space, three dimensions, and to arbitrary turn angles.  These and other considerations will make a more realistic model for future work.  

Further testing of the multi-copter UAV is indicated and will allow more extensive testing of the robustness and accuracy of the hardware design. New sensors that can identify and detect flying insects~\cite{chen2014flying} may be added to the UAV and enable it to proactively steer toward insect swarms and identify insects in realtime.

The concept may be extended to a non-destructive population survey in which the screen could be replaced with a net and, with appropriate lighting, the camera used to record capture events.  Teams of UAVs could work together to map areas more quickly and, by measuring gradients of the distribution, quickly find large mosquito populations.


%
%
%\bibliographystyle{IEEEtran}
%\bibliography{./bibs/match}
%
%% that's all folks
%\end{document}
%
%
