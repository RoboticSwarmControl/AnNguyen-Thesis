
% abstract for thesis

A wireless sensing networking (WSN) is a network of two or more nodes of sensor, each with its own data storage, processor and data transmitter.
Each node in a WSN have a smaller power requirement, each sensor have a smaller range requirement, and each node can have its own power supply.
Even compared to wired sensor network where sensor nodes have their own power supply, a WSN eliminiate wiring complications.
Currently, many sensor networks are wired sensor network, or wireless sensor with a manual maintenance, where a person have to interface with the sensor node to download, analyze data and recharge the node, or replace battery.
In this thesis, I built Unmanned Aerial Vehicles and used them to distribute WSN, as well as show how UAV can be used to monitor, download data and recharge these 


An Unmanned Vehicle, such as a flying multi-copter, an unmnanned rover, an remote-controlled boat, can be used to cover large area of land, to perform repetitive, tedius yet strainuous task for human.
We can have an Unmanned Aerial Vehicle (UAV) distribute a network of seismic microphone, used during seismic surveying, in trecherous terrain, free of heavy audio wiring, without risking injury to human worker.
We can have an UAV sweep a large area with a mosquito zapping net, destructively sampling mosquito population in the area, giving entomology researcher better data about their distribution and behavior through time and space.
We can have an Unmanned boat, or a UAV, distribute a drifting Wireless Sensor Network into a body of water.
The same, or several Unmanned Vehicles, can then monitor, recharge and finally recollect them.
The following thesis present hardware for all of the above applications, as well as software and algorithms for the Unmanned Vehicles, and sensor nodes.
